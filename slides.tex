\documentclass[aspectratio=169,xcolor={dvipsnames,svgnames,table}]{beamer}

\usepackage[utf8]{inputenc}
\usepackage{centernot}
\beamertemplatenavigationsymbolsempty
\usetheme{metropolis}
\usecolortheme{owl}
\title{Stetigkeit mit Folgenkriterium beweisen\\(Beweisschema)}
\date{}

\setbeamertemplate{footline}{}

\newcommand*{\N}{\mathbb N}
\newcommand*{\Z}{\mathbb Z}
\newcommand*{\Q}{\mathbb Q}
\newcommand*{\R}{\mathbb R}
\newcommand*{\C}{\mathbb C}

\begin{document}
  \begin{frame}
    \titlepage
  \end{frame}

  \begin{frame}
    \frametitle{Folgenkriterium: Beweis der Stetigkeit in einem Punkt}

    \uncover<2->{Sei $f:\ldots$ eine Funktion mit $f(x)=\ldots$.} \uncover<3->{Sei außerdem $x_0=\ldots$.} \uncover<4->{Sei $(x_n)_{n\in\N}$ eine beliebige Folge von Argumenten mit $\lim_{n\to\infty} x_n=x_0$.} \uncover<5->{Es gilt:}

    \begin{align*}
      \uncover<6->{\lim_{n\to\infty} f(x_n) = \ldots = f(x_0)}
    \end{align*}
  \end{frame}

  \begin{frame}
    \frametitle{Folgenkriterium: Beweis der Stetigkeit in einem Punkt}

    \uncover<2->{Sei $f:\ldots$ eine Funktion mit $f(x)=\ldots$.} \uncover<3->{Sei außerdem $x_0$ ein beliebiges Argument von $f$.} \uncover<4->{Sei $(x_n)_{n\in\N}$ eine beliebige Folge von Argumenten mit $\lim_{n\to\infty} x_n=x_0$.} \uncover<5->{Es gilt:}

    \begin{align*}
      \uncover<6->{\lim_{n\to\infty} f(x_n) = \ldots = f(x_0)}
    \end{align*}
  \end{frame}
   \begin{frame}

  \end{frame}
\end{document}
